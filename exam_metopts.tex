\documentclass[11pt,a4paper]{report}
\usepackage[a4paper, left=20mm, right=20mm, top=30mm, bottom=20mm]{geometry}
\usepackage{amsmath,amsfonts,amssymb,amsthm,epsfig,epstopdf,titling,url,array,tkz-berge}
\usepackage[T2A,T1]{fontenc}
\usepackage[utf8]{inputenc}
\usepackage[russian]{babel}
\usepackage{xparse}
\usepackage[shortlabels]{enumitem}
\usepackage{float}
\usepackage{cancel}
\usepackage{multicol}
\usepackage{listings}

\usepackage{hyperref}
\hypersetup{
	colorlinks=true,
	linkcolor=blue,
	filecolor=magenta,      
	urlcolor=cyan,
}
\def\E{\mathbb{E}}
\def\Var{\mathrm{Var}}
\def\cov{\mathrm{cov}}
\def\corr{\mathrm{corr}}
\def\salg{\mathcal{F}}
\def\P{\mathrm{P}}
\def\borel{\mathcal{B}}
\def\cantor{\mathcal{C}}

\def\eps{\varepsilon}
\def\phi{\varphi}
\def\Real{\mathbb{R}}
\def\Proj{\mathbb{P}}
\def\Hyper{\mathbb{H}}
\def\Integer{\mathbb{Z}}
\def\Natural{\mathbb{N}}
\def\Complex{\mathbb{C}}
\def\Rational{\mathbb{Q}}
\def \tr{\mbox{tr}}
\def \grad{\nabla}
\def \mse{\mbox{MSE}}
\def \etheta{\widehat{\theta}}
\def \esigma{\widehat{\sigma}}

\def\le{\leqslant}
\def\ge{\geqslant}

\usepackage{eucal}
\usepackage{stackengine,graphicx,amssymb}
\stackMath
\newcommand\frightarrow{\scalebox{1}[.3]{$\rule[.45ex]{2ex}{1.5pt}%
		\kern-.2ex{\blacktriangleright}$}}
\newcommand\darrow[1][]{\mathrel{\stackon[1pt]{\stackanchor[1pt]{\frightarrow}{\frightarrow}}{\scriptstyle#1}}}

\newcommand\independent{\protect\mathpalette{\protect\independenT}{\perp}}
\def\independenT#1#2{\mathrel{\rlap{$#1#2$}\mkern2mu{#1#2}}}
\renewcommand{\thesection}{\arabic{section}}


\theoremstyle{definition}
\newtheorem{problem}{Задача}
\newtheorem*{uproblem}{Задача}

\theoremstyle{definition}
\newtheorem{theorem}{Теорема}[section]
\newtheorem{lemma}{Лемма}[section]
\newtheorem{statement}{Утверждение}[section]
\newtheorem*{corollary}{Следствие} 
\theoremstyle{definition}
\newtheorem*{definition}{Определение}
\newtheorem{example}{Пример}[section]

\newcommand\mydots{\makebox[1em][c]{.\hfil.\hfil.}}
\renewcommand{\thesection}{\arabic{section}}

\title{Теория к зачёту по методам оптимизации}
\author{Иванов Вячеслав, группа 699}

\begin{document}
	\setlength{\parindent}{1cm}
	{\let\newpage\relax\maketitle}
	\tableofcontents
	\newpage
	\section{Безусловные методы первого порядка}
	\subsection{Градиентный спуск}
	Основная формула\footnote{Дискретизация ОДУ $ \stackrel{.}{x}(t) = -\nabla f(x) $}:
	$$
		x_{k+1} = x_{k} - \alpha_k \nabla f(x_k)
	$$
	\begin{theorem}[\textit{Выпуклая функция, оценка сверху}]$  $\\
		Пусть целевая функция $ f $ выпуклая с липшицевым градиентом, а $ \alpha = \frac{1}{L} $. Тогда:
		$$
			f(x_{k+1}) - f^{*} \le \frac{2L \| x - x_0 \|^2_2}{k+4} = \mathcal{O} \left (\frac{1}{k}\right )
		$$
		Т.е. сходимость сублинейная в смысле$ \| x_{k+1} - x^{*} \|_2 \le Ck^{\alpha},\ \alpha < 0 $.\\
		Т.е. для достижения точности $ \eps $ нужно $ \mathcal{O}\left( \frac{1}{\eps} \right) $ итераций.
	\end{theorem}
	\begin{theorem}[\textit{Сильно выпуклая функция, оценка сверху}]$ $\\
		Пусть $ f $ сильно выпукла с липшицевым градиентом, а $ \alpha = \frac{2}{L + \mu} $. Тогда:
		$$
			f(x_k) - f^{*} \le \frac{L}{2} \left ( \frac{\kappa - 1}{\kappa + 1} \right )^{2k} \| x_0 - x^* \|_2^2,\ \kappa = \frac{L}{\mu}
		$$
		Т.е. сходимость линейная в смысле определения $ \| x_{k+1} - x^{*} \|_2 \le Cq^{k}$ с $ q = \frac{\kappa - 1}{\kappa + 1} $.\\
		Т.е. для достижения точности $ \eps $ нужно $ \mathcal{O}\left( \kappa \ln \frac{1}{\eps} \right) $ итераций.
	\end{theorem}
	\begin{theorem}[\textit{Выпуклая функция, оценка снизу}]$  $\\
		$$
			f(x_{k+1}) - f^{*} \ge \frac{3 \| x_0 - x^* \|_2^2 }{32L(k+1)^2}
		$$
		т.е. показатель сублинейной сходимости $ \alpha \in (-1,\ -0.5) $.\\
		Более того, оценка верна для всех методов первого порядка.
	\end{theorem}
	\begin{theorem}[\textit{Сильно выпуклая функция, оценка снизу}]$  $\\
		$$
			f(x_{k+1}) - f^{*} \ge \frac{\mu}{2} \left (\frac{\sqrt{\kappa} - 1}{\sqrt{\kappa} + 1}\right )^{2k} \| x_0 - x^* \|_2^2 
		$$
		т.е. знаменатель прогрессии в линейной сходимости $ q \in \left (\frac{\kappa - 1}{\kappa + 1}; \frac{\sqrt{\kappa} - 1}{\sqrt{\kappa} + 1}\right ) $.
	\end{theorem}
	\subsection{Метод тяжёлого шарика}
	Основная формула\footnote{Дискретизация ОДУ второго порядка — затухающих колебаний в вязкой среде}:
	$$
		x_{k+1} = x_{k} - \alpha_k \nabla f(x_k) + \beta_k (x_k - x_{k-1})	
	$$
	\begin{theorem}[\textit{Сильно выпуклая функция, оценка сверху}]$  $\\
		Пусть $ f $ — сильно выпуклая функция с липшицевым градиентом. Тогда:
		\begin{gather*}
			\alpha_k = \frac{4}{(\sqrt{L} + \sqrt{\mu})^2},\ \beta_k = \max \{ |1 - \sqrt{\alpha_k L}|,\ |1 - \sqrt{\alpha_k \mu}|\}^2 \implies \\
			\implies \left \| \begin{bmatrix}
				x_{k+1} - x^{*}\\
				x_{k} - x^{*}
			\end{bmatrix} \right \|_{2} \le \left( \frac{\sqrt{\kappa} - 1}{\sqrt{\kappa} + 1} \right) ^ k \left \| \begin{bmatrix}
			x_{1} - x^{*}\\
			x_{0} - x^{*}
			\end{bmatrix} \right \|_{2}
		\end{gather*}
		Т.е. сходимость линейная в смысле определения $ \| x_{k+1} - x^{*} \|_2 \le Cq^{k}$ с $ q = \frac{\sqrt{\kappa} - 1}{\sqrt{\kappa} + 1} $.\\
		Т.е. для достижения точности $ \eps $ нужно $ \mathcal{O}\left( \sqrt{\kappa} \ln \frac{1}{\eps} \right) $ итераций.
	\end{theorem}
	\subsection{Метод Нестерова}
	Основные формулы:
	\begin{align*}
		x_{k+1} &= y_{k} - \alpha_k \nabla f(y_{k})\\
		y_{k+1} &= x_{k+1} + \frac{k}{k+3} (x_{k+1} - x_{k})
	\end{align*}
	\begin{theorem}[\textit{Оценка сверху}]$  $\\
		$$
			f(x_{k+1}) - f^{*} \le \frac{2L \| x_{0} - x^{*} \|_2^2 }{(k+2)^2}
		$$
		Т.е. сходимость сублинейная в смысле$ \| x_{k+1} - x^{*} \|_2 \le Ck^{\alpha},\ \alpha < 0 $.\\
		Т.е. для достижения точности $ \eps $ нужно $ \mathcal{O}\left( \frac{1}{\sqrt{\eps}} \right) $ итераций.
	\end{theorem}
	\subsection{Метод сопряжённых градиентов}
	$$
		\min \frac{1}{2} x^{\top} A x - b^{\top} x,\ A \in S^{n}_{++}
	$$
	Идея: каждым шагом устранять отличие $ x_k $ от $ x^* $ в одной координате (как минимум).\\
	Факт: для этого не обязательно знать $ x^* $. По сути, обобщается идея предобуславливания.\\
	\begin{definition}[\textit{А-ортогональность}]
		$$  
			x, y:\ x^{\top} A y = 0,\ A \in S^{n}_{++}
		$$
		Эквивалентно ортогональности относительно скалярного произведения с матрицей Грама $ A $.
	\end{definition}
	\begin{figure}[H]
		\includegraphics[width=\textwidth]{conjugate_gradients.png}
	\end{figure}
	\noindent В алгоритме выстраивается последовательность $ A $-ортогональных векторов $ p_j $, через которые пересчитываются направления $ x_k $. Формулы пересчёта имеют вид:
	\begin{gather*}
		p_{0} = -r_{0} = Ax - b\\
		x_{k+1} = x_{k} + \alpha_k p_{k} \\
		r_{k+1} = r_{k} + \alpha_k A p_k\\ 
		p_{k+1} = -r_{k+1} + \beta_{k+1} p_{k} \\
		\alpha_{k+1} = -\frac{r_{k}^{\top}p_k}{p_k^{\top} A p_k},\ \beta_{k+1} = \frac{p_k^{\top} A r_{k+1}}{p_{k}^{\top} A p_{k}}
	\end{gather*}
	Где формула для $ \beta_{k+1} $ выводится из условия $A$-ортогональности векторов $ p_{k+1} $ и $ p_{k} $, а для $ \alpha_{k+1} $ — из условия $ r_{k} \perp p_{i},\ i = 1,\ldots,k-1 $. Тогда $ x_{k} = \underset{x\ \in P}{\arg\min} f(x),\ P = x_{0} + \mathrm{span} (p_0, \ldots, p_{k-1})$\\
	Формулы пересчёта коэффициентов упрощаются:
	\begin{align*}
		\alpha_k &= -\frac{r_{k}^{\top} p_k}{p_k^{\top} A p_k} =\\
		&= -\frac{r_{k}^{\top} (-r_k + \beta_k p_{k-1})}{p_k^{\top} A p_k}=\\
		&= \frac{\| r_k \|_2^2}{p_k^{\top} A p_k} + \frac{r_k^{\top} p_{k-1}}{p_k^{\top} A p_k}= [r_k \perp p_{k-1}] =\\
		&= \frac{\| r_k \|_2^2}{p_k^{\top} A p_k}\\
		\beta_{k+1} &= \frac{r_{k+1}^{\top} A p_k}{p_k^{\top} A p_{k}} = [\alpha_k A p_k = r_{k+1} - r_{k}] =\\
		&= \frac{r_{k+1}^{\top} (r_{k+1} - r_{k})}{(-r_{k} + \beta_k p_{k-1})^{\top} (r_{k+1} - r_{k})} =\\
		&= \frac{\| r_{k+1} \|_2^2}{\| r_{k} \|_2^2}
	\end{align*}
	\begin{theorem}$  $
		\begin{itemize}[$\diamond$]
			\item $ r_k \perp r_i,\ i < k $
			\item $ \mathrm{span} (r_0, \ldots, r_k) = \mathrm{span} (p_0, \ldots, p_k) = \mathrm{span} (r_0, Ar_0, \ldots, A^k r_0) $
			\item $ p_k^{\top} A p_k = 0,\ i < k $
			\item Метод сопряжённых градиентов оптимален для квадратичной целевой функции.\\ 
			Он сходится за $ |\mathrm{spec} (A)| \le n $ итераций, где $ \mathrm{spec}(A) $ — спектр оператора A.
			\item Справедлива оценка:
			$$
				f_k - f^{*} \le C \left ( \frac{\sqrt{\kappa} - 1}{\sqrt{\kappa} + 1} \right )^{k}\\
			$$
		\end{itemize}
	\end{theorem}
	\subsection{Метод Флетчера-Ривза}
	Обобщение метода сопряжённый градиентов на неквадратичную целевую функцию.
	\begin{itemize}[$\diamond$]
		\item $ \alpha_k $ подбирается адаптивно (замкнутой формы нет)
		\item $ \beta_k $ ищется с помощью градиентов $ \nabla f'(x_{k-1}), \nabla f'(x_{k-2}) $
		\item $ r_k $ в формулах заменяется на $ \nabla f'(x_k) $ и пересчитывается в лоб
	\end{itemize}
	При этом:
		\begin{itemize}[$\diamond$]
		\item С ростом числа итераций направления $ p_k $ становятся всё более коллинеарными.
		\item Не при любом способе выбора $ \alpha_k $ получается направление убывания.
	\end{itemize}
	Первая проблема решается рестартами, вторая разобрана не была.
	\section{Проксимальные методы}
	\subsection{PGM и его частные случаи}
	\begin{definition}[\textit{Проксимальный оператор}]
		$$
			\underset{\alpha f}{\mathrm{prox}}(x) = \underset{u}{\arg\min} \left (f(u) + \frac{1}{2\alpha}\| x - u \|_2^2\right )
		$$
		Если $ f $ выпукла, то проксимальный оператор — сильно выпуклая функция.
	\end{definition}
	\begin{theorem}
		Минимум функции будет неподвижной точкой проксимального оператора.
	\end{theorem}
	\begin{statement}
		Для проксимального оператора выполнено некоторое подобие линейности:
		$$
			f(x) = \sum\limits_{i=1}^{n} f_{i}(x_i) \implies \underset{f}{prox}(v)_{i} = \underset{f_i}{prox}(v_i) 
		$$
	\end{statement}
	\begin{definition}[\textit{Проксимальный градиентный метод}]$  $\\
		Пусть выпуклая функция $ f $ представима в виде $ f = g + h $, где $ g $ выпукла и может принимать бесконечные значения, а $ h $ дифференцируемая и выпуклая. Тогда положим:
		\begin{align*}
			x_{k+1} &= \underset{\alpha_k g}{\mathrm{prox}} (x_k - \alpha_k \nabla h(x_k)) =\\
							&= \underset{x}{\arg\min} \left (g(x) + \frac{1}{2\alpha_k} \| x - (x_k - \alpha_k \nabla h(x_k)) \|_2^2 \right ) =\\
							&=  \underset{x}{\arg\min} \left (g(x) + h(x_k) + \langle \nabla h(x_k), x - x_k \rangle + \frac{1}{2\alpha_k} \| x - x_k \|_2^2 \right )
		\end{align*}
		Такой метод сходится как $ \mathcal{O}\left ( \frac{1}{k} \right ) $ для $ \alpha_k \equiv \mathrm{const} \in \left (0, \frac{1}{L}\right ] $, где $ L $ — константа Липшица для $ \nabla h $.\\
		Видно, что:
		\begin{enumerate}
			\item Если $ g(x) := \begin{cases}
				x,& x \in G\\
				+\infty, &\text{otherwise}
			\end{cases} $ — индикатор выпуклого множества $ G $, — то:
			\begin{align*}
				&\underset{x}{\arg\min} \left (g(x) + h(x_k) + \langle \nabla h(x_k), x - x_k \rangle + \frac{1}{2\alpha_k} \| x - x_k \|_2^2 \right ) =\\
				&= \underset{x \in G}{\arg\min} \left ( h(x_k) + \langle \nabla h(x_k), x - x_k \rangle + \frac{1}{2\alpha_k} \| x - x_k \|_2^2 \right )
			\end{align*}
			— т.н. \textit{метод проекции градиента}.
			\item Если $ h(x) \equiv 0 $, то получаем \textit{проксимальный метод}.
			\item Если $ g(x) \equiv 0 $, то получаем градиентный спуск (при $ \alpha_k = \frac{1}{L} $).
		\end{enumerate}
		Сходимость каждого метода аналогична оной у градиентного спуска (с поправкой на то, что в методе проекции градиента нужно учесть сложность вычисления проекции точки на множество). 
	\end{definition}
	\begin{theorem}[\textit{Метод проекции градиента, оценка сверху}]$  $\\
		Если $ f $ выпуклая с липшицевым градиентом на замкнутом выпуклом мн-ве $ P $, то при $ \alpha \in \left (0, \frac{2}{L} \right ] $:		
		\begin{itemize}[$\diamond$]
			\item $ x_k \to x^{*} $ сублинейно, а в случае сильной выпуклости — линейно со знаменателем $ q $
			\item Если при этом $ \exists l > 0 \forall x \in P: \nabla^2 f(x) \succeq l \textbf{I} $, то $ q = \max \{ |1 - \alpha l|, |1 - \alpha L| \} $
		\end{itemize}
	\end{theorem}
	\begin{definition}[\textit{FISTA}\footnote{Fast Iterative Shrinkage-Thresholding Algorithm}]
		Аналог метода Нестерова для проксимального градиентного метода:
		\begin{align*}
			y_{1} &:= x_{0},\ t_{1} := 1,\ k = 1\\
			x_{k} &:= \underset{\alpha_k g}{\mathrm{prox}}(y_k - \alpha_k \nabla h(y_k))\\
			t_{k+1} &:= \frac{1 + \sqrt{1 + 4t_k^2}}{2}\\
			y_{k+1} &:= x_{k} + \frac{t_k - 1}{t_{k+1}}(x_k - x_{k-1})
		\end{align*}
	\end{definition}
	\begin{theorem}[\textit{О сходимости FISTA}]$  $\\
		\bgroup
		\def\arraystretch{1.5}
		\begin{tabular}{c|c|c|c}
			\hline 
			& шаг & скорость сходимости & число итераций\\
			\hline
			$ h $ выпуклая с липшицевым градиентом & $ \alpha_k \equiv \frac{1}{k} $ & $ \mathcal{O}\left (  \frac{1}{k^2} \right ) $ & $ \mathcal{O}\left (  \frac{1}{\sqrt{\eps}} \right ) $  \\
			\hline
			$ h $ сильно выпуклая с липшицевым градиентом & $ \alpha_k \equiv \frac{1}{k} $ & $ \mathcal{O}\left ( \left ( 1 - \frac{1}{\sqrt{\kappa}} \right )^{k} \right ) $ & $ \mathcal{O}(\sqrt{\kappa} \log \frac{1}{\eps}) $ \\
			\hline 
		\end{tabular}
		\egroup
	\end{theorem}
	\noindent \textbf{Pro:}
	\begin{itemize}[$\diamond$]
		\item Часто проекцию можно вычислить аналитически
		\item При этом сходимость аналогична градиентому спуску в безусловной оптимизации
		\item Обобщается на негладкий случай
	\end{itemize}
	\textbf{Contra:}
	\begin{itemize}[$\diamond$]
		\item Проекция не всегда вычисляется за $ \mathcal{O}(1) $ (пример — проекция на политоп)
		\item При обновлении градиента может меняться структура задачи (из-за свойств множества)
	\end{itemize}
	\section{Условные методы первого порядка}
	\subsection{Метод Франка-Вольфе (условного градиента)}
	Идея: в методе градиентного спуска подбирать направление и длину шага так, чтобы не выходить за пределы множества. Это возможно, если ограничения задают выпуклое замкнутое множество.\\
	$$
		f(x_k + s_k) \approx f(x_k) + \langle \nabla f(x_k), s_k \rangle \to \min_{s_k \in P}
	$$	
	Направление $ s_k - x_k $ называется \textit{условным градиентом} функции $ f $ в точке $ x_k $ на мн-ве $ P $.
	В силу выпуклости, $ f(s) \ge f(x_k) + \langle \nabla f(x_k), s - x_k \rangle $, откуда получаем аналог зазора двойственности:
	$$
		f(x) - f(x^*) \le -\min_{s \in P} \langle \nabla f(x_k), x - s \rangle = \max_{s \in P} \langle \nabla f(x_k), x - s \rangle = g(x)
	$$
	\begin{theorem}[\textit{Метод условного градиента, оценка сверху}]$  $\\
		Пусть $ f $ выпуклая с липшицевым градиентом на выпуклом компакте $ X $. Тогда при $ \alpha_k = \frac{2}{k+1} $:
		$$
			f(x_k) - f(x^*) \le \frac{2d^2L}{k+2},\ k \ge 1,\ d = \mathrm{diam}(X)
		$$
		Полученная сублинейная скорость сходимости не зависит от размерности. Тем не менее, она неулучшаема даже для сильно выпуклых функций, а метод не обобщается на негладкий случай.
	\end{theorem}
	\section{Методы второго порядка}
	\subsection{Метод Ньютона}
	Основная формула: 
	$$
		x_{k+1} = x_{k} - (\nabla^2 f(x_k))^{-1} \nabla f(x_{k})
	$$
	Выводится из квадратичной аппроксимации при $ \nabla^2 f(x) \succ 0 $:
	\begin{align*}
		f(x + h) &\approx \left. f(x) + \nabla f(x)^{\top} h + \frac{1}{2} h^{\top} \nabla^2 f(x) h\ \middle|\ \frac{\partial}{\partial h} \right.  \\
		0 &= \nabla f(x) + \nabla^2 f(x) h\\
		h &= -(\nabla^2 f(x))^{-1} \nabla f(x)
	\end{align*}
	Аналогичный метод применяется для решения систем нелинейных уравнений:
	\begin{align*}
		G(x) &= 0,\quad G : \Real^{n} \to \Real^{n}\\
		G(x_k + \Delta x) &\approx G(x_k) + \nabla G(x_k) \Delta x = 0\\
		\Delta x &= -(\nabla G(x_k))^{-1} G(x_{k})\\
		x_{k+1} &= x_{k} - (\nabla G(x_k))^{-1} G(x_{k})
	\end{align*}
	Можно заметить, что необходимость решать такие системы возникает из условий оптимальности:
	$$
		f'(x^{*}) = G(x) = 0	
	$$
	\begin{definition}[\textit{Демпфированный метод Ньютона}]
		$$
			x_{k+1} = x_{k} - \alpha_k (\nabla^2 f(x_k))^{-1} \nabla f(x_k)
		$$
		Адаптивный выбор размера шага расширяет область сходимости.
	\end{definition}
	\begin{theorem}[\textit{Характер сходимости в методе У а}]$  $\\
		Пусть 
		\begin{enumerate}
			\item $ f(x) $ локально сильно выпукла с константой $ \mu $: $ \exists x^{*}: \nabla^2 f(x^*) \succeq \mu I $ 
			\item Её гессиан липшицев с константой $ M $: $ \| \nabla^2 f(x) - \nabla^2 f(y) \| \le M \| x - y \| $
			\item Начальная точка достаточно близка к оптимальной: $ \| x_0 - x^* \| \le \frac{2\mu}{3M} $
		\end{enumerate}
		Тогда метод Ньютона сходится квадратично:
		$$
			\| x_{k+1} - x^* \| \le \frac{M \| x_k - x^* \|^2}{2(\mu - M \| x_k - x^* \|)}
		$$
		Т.е. для достижения точности $ \eps $ нужно $ \mathcal{O} \left (\log\log \frac{1}{\eps}\right ) $ итераций.
	\end{theorem}
	\subsection{Квазиньютоновские методы}
	Мотивированы тем, что в методе Ньютона нужно явно хранить гессиан ($ \mathcal{O}(n^2) $ памяти) и обращать его ($ \mathcal{O}(n^3) $ операций) на каждой итерации. Кроме того, на очередной итерации он может оказаться вырожденным. Этих недостатков можно избавиться, если заменить гессиан на его приближение и придумать разумные формулы пересчёта. Посмотрим на две крайности:
	\begin{itemize}[$\diamond$]
		\item \textbf{Градиентный спуск:}
		\begin{align*}
			f(x + h) &\le f(x) + \langle \nabla f(x), h \rangle + \frac{1}{2\alpha} h^{\top} \textcolor{red}{I} h,\ \alpha \in (0, L^{-1}]\\
			\min_{h} f_{g}(h) &\implies h^{*} = -\alpha\nabla f(x)\\
			x_{k+1} &= x_{k} - \alpha_k \nabla f(x_k)
		\end{align*}
		\item \textbf{Метод Ньютона:}
		\begin{align*}
			f(x + h) &\approx f(x) + \langle \nabla f(x), h \rangle + \frac{1}{2\alpha} h^{\top} \textcolor{red}{\nabla^2 f(x)} h,\ \nabla^2 f(x) \succ 0 \\
			\min_{h} f_{N}(h) &\implies h^{*} = -(\nabla^2 f(x))^{-1} \nabla f(x)\\
			x_{k+1} &= x_{k} - (\nabla^2 f(x_k))^{-1} \nabla f(x_k) 
		\end{align*}
	\end{itemize}
	Все квазиньютоновские методы находятся "между ними":
	\begin{align*}
		f_{q}(h) := f(x + h) &\approx f(x) + \langle \nabla f(x), h \rangle + \frac{1}{2\alpha} h^{\top} \textcolor{red}{B_k} h,\ B_k \succ 0\\
		\min_{h} f_{q}(h) &\implies h^{*} = -B_k^{-1} \nabla f(x) \\
	x_{k+1} &= x_{k} - B_k^{-1} \nabla f(x_k) =\\
					&= x_{k} - H_k \nabla f(x_k)
	\end{align*}
	К $ B_k $, кроме близости к гессиану по матричной норме, предъявляются следующие требования:
	\begin{itemize}[$ \diamond $]
		\item Быстрый пересчёт $ B_k \to B_{k+1} $, когда доступны только градиенты.
		\item Быстрый поиск направления $ h_k $.
		\item Возможность компактного хранения $ B_k $.
		\item Сверхлинейная сходимость.
	\end{itemize}
	Для обновления $ B_k $ есть следующие 2 правила:
	\begin{enumerate}
		\item \textbf{Правило двух градиентов:} 
			\begin{align*}
				f'_{q}(-\alpha_k h_k) &= f'(x_k) \implies \nabla f(x_{k+1}) - \alpha_k B_{k+1} h_k = \nabla f(x_k) \\
				f'_{q}(0) &= f'(x_{k+1})
			\end{align*}
		\item \textbf{Secant equation:}
		\begin{align*}
			s_{k} &:= x_{k+1} - x_{k}\\
			y_{k} &:= \nabla f(x_{k+1}) - \nabla f(x_k)\\
			B_{k+1} s_k &= y_k
		\end{align*}
	\end{enumerate}
	\subsection{Barzilai-Borwein}
	Аппроксимация гессиана диагональной матрицей:
	$$
		\alpha_k \nabla f(x_k) = \alpha_k I \nabla f(x_k) = \left ( \frac{1}{\alpha_k} I \right )^{-1} \nabla f(x_k) \approx (\nabla^2 f(x))^{-1} \nabla f(x)
	$$
	Квазиньютоновское уравнение принимает вид:
	$$
		\alpha_k^{-1} s_{k-1} \approx y_{k-1}
	$$
	Отсюда естественным образом возникает задача наименьших квадратов:
	$$
		\min_{\alpha_k} \frac{1}{2} \| s_{k-1} - \alpha_{k} y_{k-1} \|_2^2 \implies \alpha_k = \frac{s_{k-1}^{\top} y_{k-1}}{y_{k-1}^{\top} y_{k-1}}
	$$
	Хотя её можно поставить иначе (скажем, в другой норме) и получить другие методы.
	\subsection{DFP}
	Поиск $ B_{k+1} $ сам по себе записывается в форме задачи оптимизации:
	\begin{align*}
		B_{k+1} = &\underset{B}{\arg\min} \| B_k - B \|_2^2\\
							&\text{s.t. } 
							\begin{aligned} 
								B^{\top} &= B\\ 
								B s_k &= y_k\\
								B &\succeq 0
							\end{aligned}
	\end{align*}
	Замкнутый вид решения, откуда $ H_{k+1} $ получается по формуле ШМВ:
	\begin{align*}
		B_{k+1} &= (I - \rho_k y_k s_k^\top) B_k (I - \rho_k s_k y_k^{\top}) + \rho_k y_k y_k^{\top},\ \rho_k = \frac{1}{y_k^\top s_k}\\
		B_{k+1}^{-1} &= H_{k+1} = H_k - \frac{H_k y_k y_k^{\top} H_k}{y_k^\top H_k y_k} + \frac{s_k s_k^\top}{y_k^\top s_k}
	\end{align*}
	\subsection{(L-)BFGS}
	Аналогичная задача оптимизации ставится сразу для $H_{k+1}$:
	\begin{align*}
		H_{k+1} = &\underset{H}{\arg\min} \| H_k - H \|_2^2\\
		&\text{s.t. } 
		\begin{aligned} 
		H^{\top} &= H\\ 
		H y_k &= s_k\\
		H &\succeq 0
		\end{aligned}
	\end{align*}
	Замкнутый вид решения, откуда $ H_{k+1} $ получается по формуле ШМВ:
	$$
		H_{k+1} = (I - \rho_k s_k y_k^\top) H_k (I - \rho_k y_k s_k^{\top}) + \rho_k s_k s_k^{\top},\ \rho_k = \frac{1}{y_k^\top s_k}
	$$
	\begin{theorem}
		Если $ f $ сильно выпуклая с липшицевым гессианом, то при некоторых дополнительных технических допущениях BFGS сходится сверхлинейно.
	\end{theorem}
	\noindent Если размерность велика, хранить эти матрицы целиком становится непрактично.\\
	Т.е. нужна процедура эффективного умножения на вектор $ \nabla f(x) $.\\
	Также есть проблема, что значения $ s_k, y_k $ на первых итерациях могут портить оценку $ B_k, H_k $ на более поздних итерациях.	Модификация L-BFGS позволяет рекурсивно вычислять $ H_{k+1} \nabla f(x_k) $ за счёт хранения только последних $ m \ll n $ значений $ (s_k, y_k) $ и обновления $ H_{m,0} $. 
	\begin{figure}[H]
		\includegraphics[width=\textwidth]{L-BFGS.png}
	\end{figure}
	\noindent Отсюда получаем $ \mathcal{O}(n^2) $ операций на одну итерацию и линейную сложность по памяти.\\
	Тем не менее, у метода есть недостатки:
	\begin{itemize}[$\diamond$]
		\item Он не рандомизируется.
		\item Нужно подбирать $ B_0 $ или $ H_0 $.
		\item Для него нет разработанной теории сходимости и точных оценок.
		\item Не любой способ выбора шага гарантирует, что $ y_k^\top s_k > 0 $.
	\end{itemize}
	\section{Линейное программирование (LP)}
	\begin{align*}
		\min_{x}&\ c^\top x\\
		\text{s.t. }& Ax \preceq 0,\\
		& x \succeq 0
	\end{align*}
	\begin{definition}[\textit{Угловая точка}]$  $
		\begin{enumerate}
			\item Tочка из допустимого множества называется \textit{вершиной многоугольника}, если она не лежит на отрезке между двумя другими точками многоугольника.
			\item Точка $x$ называется \textit{угловой точкой} многоугольника, если:
			\begin{itemize}[$\diamond$]
				\item Она лежит в множестве.
				\item Существует такое множество $\mathcal{B} \subset \{1, \dots, n \}$, что 
				\begin{itemize}
			  	\item $|\mathcal{B}| = m$
			  	\item $i \notin \mathcal{B} \Rightarrow x_i = 0$
			  	\item матрица $B = [a_i]_{i \in \mathcal{B}}$ невырождена, где $a_i$ - $i$-ый столбец матрицы $A$.
			   \end{itemize}
		   Матрица $B$ называется \textit{матрицей базиса}.
			\end{itemize}
		\end{enumerate}
	\end{definition}
	\begin{theorem}
		Все угловые точки являются вершинами многоугольника $ Ax \preceq b $.
	\end{theorem}
	\begin{theorem}[\textit{Основная теорема линейного программирования}]$  $
		\begin{itemize}[$\diamond$]
			\item Если в задаче линейного програмирования допустимое множество непусто, тогда оно имеет как минимум одну угловую точку.
			\item Если задача линейного программирования имеет решения, тогда хотя бы одно из них является угловой точкой.
			\item Если задача линейного программирования ограничена и допустимое множество непусто, тогда она имеет оптимальное решение.
		\end{itemize}
	\end{theorem}
	\noindent Основной алгоритм решения задачи линейного программирования называется \textit{симплекс-методом}:
	\begin{enumerate}
		\item Начальная точка — произвольная вершина.
		\item Перебрать смежные вершины и пойти в ту, в которой значение целевой функции меньше.
	\end{enumerate}
	\textbf{Шаги алгоритма:}\\
	Дана угловая точка $x$, соответствующая ей матрица базиса $B$ и множество индексов $\mathcal{B}$.
	\begin{enumerate}
		\item Вычислить \textit{оценки замещения} (reduced costs) $\overline{c}_j = c_j - c^{\top}_{\mathcal{B}}B^{-1}a_j$ для всех $j \not\in \mathcal{B}$.
		\begin{enumerate}
			\item если $\overline{c}_j \geq 0$ для всех $j$, то текущее значение является оптимальным и уменьшить целевую функцию нельзя
			\item иначе $\textcolor{red}{\text{выбрать}} $ индекс $j^*$, для которого $\overline{c}_{j^*} < 0$
		\end{enumerate}
		\item Вычислить $u = B^{-1}a_{j^*}$
		\begin{enumerate}
			\item если все компоненты $u$ неположительны, то задача неограничена, оптимальное значение равно $-\infty$
			\item если есть положительные компоненты, то 
			$$
				\theta^* := \min_{\{i | u_i > 0\}} \frac{x_{\mathcal{B}(i)}}{u_i}
			$$
		\end{enumerate}
		\textbf{По смыслу} — выбрать столбец в симплекс-таблице на основе оценок замещения.
		\item  Пусть $\ell$ $\textcolor{red}{\text{такой}} $ индекс, что 
		$$
			\theta^* = \frac{x_{\mathcal{B}(\ell)}}{u_{\ell}}.
	  $$
	  Сформировать новую матрицу базиса $\hat{B}$ с помощью замены столбца $a_{\mathcal{B}(\ell)}$ на столбец $a_{j^*}$. Новая угловая точка $\hat{x}$, соответствующая матрице базиса $\hat{B}$, определяется так:
    \begin{align*}
      & \hat{x}_{j^*} = \theta^*\\
      & \hat{x}_{\mathcal{B}(k)} = x_{\mathcal{B}(k)} - \theta^*u_k, \text{если } k \neq \ell
    \end{align*}
    \textbf{По смыслу} — выбрать строку в симплекс-таблице и провести итерацию метода Гаусса с ведущим элементом на пересечении выбранных строки и столбца.
	\end{enumerate}
	Чтобы избежать зацикливания, которое может произойти, если угловая точка содержит больше $ (m - n) $ нулевых координат, на каждом шаге нужно выбирать лекс. минимальные индексы.\\
	\begin{itemize}[$\diamond$]
		\item Было показано, что в худшем случае время работы симплекс-метода \textbf{экспоненциально} зависит от размерности задачи!
		\item Однако на практике сложность чаще всего пропорциональна количеству ограничений и симплекс-метод сходится быстро.
		\item Почему это так, неясно до сих пор.
	\end{itemize}
	Занимательный факт: в невырожденных задачах линейного программирования всегда выполнена сильная двойственность. При этом в двойственной задаче число переменных совпадает с числом ограничений в прямой задаче, что может сильно понизить размерность. Но неясно, как по двойственным переменным восстановить ответ прямой задачи, в нашем курсе это не обсуждалось.
	\subsection{Двухфазный симплекс-метод}
	\noindent \textbf{Проблема:} выбор начальной точки неочевиден.\\
	\textbf{Решение:} вспомогательная. задача оптимизации (при условии $ b \succeq 0 $)\footnote{А если это не так? Что тогда? Всегда ли есть сведение к такому виду?}:
	\begin{align*}
		& \min_{z, y} y_1 + \ldots + y_m \\
		\text{s.t. } & Az + y = b\\
		& z \geq 0, \; y \geq 0
	\end{align*}
	Начальная точка для такой задачи очевидна — $ z = 0,\ y = b $.
	\begin{itemize}[$\diamond$]
		\item Если оптимальное значение функции в этой задаче \textbf{не равно} $0$, то допустимое множество исходной задачи пусто.
		\item Если оптимальное значение функции в этой задаче \textbf{равно} $0$, то $y^* = 0$ и $x_0 = z^*$.
	\end{itemize}
	\subsection{М-метод}
	\textbf{Идея: } объединить двухфазный симплекс-метод в однофазный
	\begin{align*}
		& \min_{z, y} c^{\top}z + M(y_1 + \ldots + y_m) \\
		\text{s.t. } & Az + y = b\\
		& z \geq 0, \; y \geq 0
	\end{align*} 
	$M$ - произвольное большое положительное число.\footnote{Подозрительно похоже на регуляризацию.}
	\section{Полуопределённое программирование (SDP)}
	Постановка задачи полуопределённого (semidefinite) программирования:
	\begin{align*}
		\min&\ c^{\top} x\\
		\text{s.t. }& F_0 + \sum_{i=1}^{m} x_i F_i \succeq 0\\
		&F_i \in \Real^{n \times n}
	\end{align*}
	К такой постановке сводятся задачи LP:
		\begin{equation*}
		\begin{aligned}[c]
			\min&\ c^{\top} x\\
			\text{s.t. }& Ax + b \succeq 0
		\end{aligned}
		\quad \longrightarrow \quad
		\begin{aligned}[c]
			\min&\ c^{\top} x\\
			\text{s.t. }& \mathrm{diag}(b) + \sum_{i=1}^{n} x_i \mathrm{diag}(A_{:, i}) \succeq 0
		\end{aligned}
	\end{equation*}
	Тем не менее, SDP более выразительно, чем LP: 
	\begin{equation*}
		\begin{aligned}
			\min_{x}& \frac{\langle c, x \rangle^2}{\langle d, x \rangle}\\
			\text{s.t. }& Ax + b \succeq 0
		\end{aligned} \quad\longrightarrow\quad
		\begin{aligned}
			\min_{x}&\ t\\
			\text{s.t. }& Ax + b \succeq 0,\\
			&\frac{\langle c, x \rangle^2}{\langle d, x \rangle} \le t,\\
			&t \ge 0
		\end{aligned} \quad\longrightarrow\quad
		\begin{aligned}
			\min_{x}&\ t\\
			\text{s.t. }& \begin{bmatrix}
				\mathrm{diag}(Ax - b) &0 &0\\
				0 & t & c^{\top} x\\
				0 & c^{\top} x & d^{\top} x
			\end{bmatrix} \succeq 0
		\end{aligned} 
	\end{equation*}
	\noindent Такую задачу в форме LP уже не записать.\\
	Важным инструментом при сведении задачи к такой форме является дополнение по Шуру:
	\begin{statement}[\textit{Критерий положительной полуопределённости}]$  $\\
		Пусть дана блочно-диагональная матрица $M$:
		$$
			M = \begin{bmatrix} 
				A & B\\ C & D
			\end{bmatrix}
		$$
		Дополнением по Шуру блока $ D $ называется матрица $ A - BD^{-1}C $.\\
		Аналогично для блока $ A $: $ D - CA^{-1}B $.\\
		Тогда верно, что: симметричная матрица $ M \succeq 0 \iff D \succeq 0,\ A - BD^{-1}B^{\top} \succeq 0$.
	\end{statement}
	\subsection{Двойственность в SDP-задачах}
	\begin{statement}[\textit{О виде двойственной задачи}]
		$$
			\begin{aligned}
				\min&\ c^{\top} x\\
				\text{s.t. }& F_0 + \sum_{i=1}^{m} x_i F_i \succeq 0
			\end{aligned} \quad\longrightarrow\quad 
			\begin{aligned}
				\max&\ -\tr (F_0 Z)\\
				\text{s.t. }& \tr (F_i Z) = c_i\\
				& Z \succeq 0 
			\end{aligned}
		$$
	\end{statement}
	\begin{proof}
		Вид Лагранжиана следует из условий ККТ:
		\begin{align*}
			L(x, Z) &= c^\top x - \tr\left (Z^{\top}\left (F_0 + \sum_{i=1}^{m} x_i F_i \right )\right ) =\\
			&= c^\top x - \tr (Z^\top F_0) - \sum_{i=1}^{m} x_i \cdot \tr\left (Z^{\top} F_i \right ) \\
			\nabla_{x} L(x, Z) &= c - \begin{bmatrix}
				\tr\left (Z^{\top} F_1 \right ) &
				\ldots &
				\tr\left (Z^{\top} F_m \right )
			\end{bmatrix}^{\top} = 0 \\&\iff \tr\left (Z^{\top} F_i \right ) = c_i
		\end{align*}
		Отсюда:
		\begin{align*}
			L(x^*, Z) &= \inf\limits_{x} \left (c^\top x - \tr (Z^\top F_0) - \sum_{i=1}^{m} x_i \cdot \tr\left (Z^{\top} F_i \right )\right ) =\\
			&= - \tr (Z^\top F_0) =: g(Z),\ Z \succeq 0
		\end{align*}
		Итоговый вид двойственной задачи: 
		$$
			\begin{aligned}
				\max&\ -\tr (F_0 W)\\
				\text{s.t. }& \tr (F_i W) = c_i\\
				& W \succeq 0 
			\end{aligned}
		$$
		где $ W := Z^\top $. 
	\end{proof}
	\noindent Сильной двойственности может не быть (хуже, чем LP), но зазор двойственности всегда конечен.
	\begin{theorem}[\textit{О сильной двойственности в SDP}]$  $\\
		Для SDP-задачи выполнена сильная двойственность, если хотя бы одно из условий выполнено:
		\begin{enumerate}[$\diamond$]
			\item Прямая задача строго допустима, т.е. $ \exists x: F(x) \succ 0 $.
			\item Двойственная задача строго допустима, т.е. $ \exists Z \in S^n_{++} \forall i = 1,\ldots,m: \tr(ZF_i) = c_i $.
		\end{enumerate}
	\end{theorem}
	\subsection{Минимизация спектрального радиуса}
	Пусть $ A(x) = A_0 + \sum_{i=1}^{m} x_i A_i,\ \forall j > 0: A_j = A_j^\top $ и поставлена задача минимизации:
	$$
		\min_{x} \lambda_{\max}(A(x))
	$$
	может быть переписана в SDP-виде:
	\begin{align*}
		\min_{t, x}&\ t\\ 
		\text{s.t. }& tI - A(x) \succeq 0
	\end{align*}
	Если матрица несимметрична/неквадратна, то минимизируется макс. сингулярное значение:
	$$
		\min_{x} \sigma_{\max}(A(x))
	$$
	что в SDP-терминах записывается как:
	\begin{align*}
		\min_{t, x}&\ t\\ 
		\text{s.t. }& 
			\begin{bmatrix} 
				tI & A(x)\\
				A^\top(x) & tI
			\end{bmatrix} \succeq 0
	\end{align*}
	\subsection{Поиск описанного эллипсоида минимального объёма}
	Пусть дан набор точек в $ \Real^d $ и нужно найти описанный вокруг них эллипс минимальной площади.\\
	Здесь нужно воспользоваться тем, что эллипс можно задать аффинным преобразованием:
	$$
		\{ x\ |\ \| x \|_2^2 \le 1 \} \to \{ u\ |\ \| u \|_2^2 \le 1,\ u = Ax + b \}
	$$
	Площадь при этом увеличивается в $ \det(A^{-1}) $ раз, откуда возникает задача минимизации:
	\begin{align*}
		\min_{A, b}& -\log \det (A)\\
		\text{s.t. }& \| Ax_i + b \|_2^2 \le 1,\\
								& A \succ 0
	\end{align*}
	А уже ограничение на норму переписывается в SDP-виде:
	$$
		\| Ax_i + b \|_2^2 \le 1 \iff 
		\begin{bmatrix}
			I & Ax_i + b\\
			(Ax_i + b)^{\top} & 1 
		\end{bmatrix} \succeq 0
	$$
	\subsection{Приближённое решение невыпуклых задач}
	В курсе рассматривалась только задача вида:
	\begin{align*}
		\min& f_0(x)\\
		\text{s.t. }& f_i(x) \le 0\\
	\end{align*}
	где $ f_i(x) \le x^{\top} A_i x + 2 b_i^\top x + c_i $, но матрицы $ A_i $ неопределены.
	\begin{enumerate}
		\item Перепишем ограничения в матричном виде:
		$$
			f_i(x) = 
			\begin{bmatrix}
				x & 1
			\end{bmatrix}^{\top}
			\begin{bmatrix}
				A_i & b_i\\
				b_i^\top & c_i
			\end{bmatrix}
			\begin{bmatrix}
				x \\ 1
			\end{bmatrix} \le 0
		$$ 
		\item Пусть $x$ допустимая точка в исходной задаче, а $t, \tau_1, \ldots, \tau_M$ допустимые точки для введённой задаче, тогда можно выписать SDP-задачу:
		\begin{align*}
			& \max t\\
			\text{s.t. } & \tau_i \geq 0\\
			& \begin{bmatrix} A_0 & b_0 \\ b_0^{\top} & c_0 - t \end{bmatrix} + \tau_1 \begin{bmatrix} A_1 & b_1 \\ b_1^{\top} & c_1 \end{bmatrix} + \ldots + \tau_M \begin{bmatrix} A_M & b_M \\ b_M^{\top} & c_M \end{bmatrix} \succeq 0
		\end{align*}
		\begin{align*}
			& 0 \leq \begin{bmatrix} x \\ 1 \end{bmatrix}^{\top} \left(\begin{bmatrix} A_0 & b_0\\ b_0 & c_0 - t \end{bmatrix} + \tau_1 \begin{bmatrix} A_1 & b_1\\ b^{\top}_1 & c_1 \end{bmatrix} \ldots + \tau_M \begin{bmatrix} A_M & b_M \\ b_M^{\top} & c_M \end{bmatrix}\right) \begin{bmatrix} x \\ 1 \end{bmatrix} = \\
			& 0 \leq f_0(x) - t + \tau_1 f_1(x) + \ldots + \tau_Mf_M(x) \leq f_0(x) - t 
		\end{align*}
		Отсюда $ f_{0}(x) \ge t $, т.е. $\max t = \min f_{0}(x)$, задачи эквивалентны.\\
		В частности, такой подход позволяет решать задачу о пересечении эллипсоидов в $ \Real^n $.
	\end{enumerate}
	\subsection{Реласкация задачи MAXCUT}
	\textbf{Задача:} найти макс.разрез в неориентированном графе $ G = (V, E) $ с весами рёбер $ w_{ij} $.
	\begin{align*}
		\max& \frac{1}{4} \sum_{i,j=1}^{n} (1 - x_i x_j) w_{ij}\\
		\text{s.t. }& x_{i} \in \{-1, 1\}
	\end{align*}
	У этой задачи есть SDP-релаксация: 
	$$
		X = xx^{\top} \to X \succeq xx^\top,\ X_{ii} = 1
	$$
	\begin{align*}
		\max& \frac{1}{4} \sum_{i,j=1}^{n} (1 - X_{ij}) w_{ij}\\
		\text{s.t. }& X_{ii} = 1\\
								& X \succeq 0
	\end{align*}
	Пусть $ X = V^{\top} V $, а $ r $ — случайный вектор, равномерно распределённый на сфере.\\
	Если $ S = \{ i\ |\ r^\top v_i \ge 0 \} $, то 
	$$
		\E (\text{разрез}) \ge 0.876 \cdot MAXCUT
	$$
	В целом, SDP часто применяется в комбинаторной оптимизации.
	\section{Методы внутренней точки}
	\subsection{Задача выпуклой оптимизации с ограничениями типа равенств}
	\begin{align*}
		\min& f(x)\\
		\text{s.t. }& Ax = b
	\end{align*}
	где $f$ — выпукла и дважды дифф-ма, $ A \in \Real^{p \times n},\ \mathrm{rk} (A) = p < n $.\\
	Двойственная задача:
	\begin{align*}
		g(\mu) &= -b^\top \mu + \inf_{x}(f(x) + \mu^\top A x) =\\
					 &= -b^\top \mu - \sup_{x}((-A^\top \mu)^\top x - f(x)) =\\
					 &= -b^\top \mu - f^{*}(-A^\top \mu)
	\end{align*}
	\begin{align*}
		\max_{\mu}&\ -b^\top \mu - f^*(-A^\top \mu)\\
		\text{s.t. }& \mu \succeq 0
	\end{align*}
	Условия оптимальности прямой задачи при этом имеют вид:
	\begin{gather*}
		Ax^* = b,\quad \nabla f(x^*) + A^\top \mu^* = 0\\
		\begin{bmatrix}
			\nabla f & A^\top\\
			A & 0
		\end{bmatrix} \begin{bmatrix}
			x^*\\
			\mu^* 
		\end{bmatrix} = 
		\begin{bmatrix}
			0\\
			b
		\end{bmatrix}
	\end{gather*}
	Такую систему можно решать методом Ньютона, если $ f $ дважды дифференцируема с липшицевым гессианом и ограничена снизу. Причём часто минимизируют квадратичную аппроксимацию:
	\begin{align*}
		\min_{v}&\ f(x) + \nabla f(x)^{\top} v + \frac{1}{2} v^{\top} \nabla^2 f(x) v\\
		\text{s.t. }& A(x + v) = b
	\end{align*}
	Условия оптимальности для такой задачи выгодно линеаризуются:
	\begin{align*}
		A(x + v^*) &= b \iff Av^* = 0\\
		\nabla f(x + v^*) + A^\top w &\approx \nabla f(x) + \nabla^2 f(x)^{\top} v^* + A^\top w = 0\\
		&\implies \nabla^2 f(x)^{\top} v^* + A^\top w = -\nabla f(x)
 	\end{align*}
 	Отсюда получаем систему:
 	$$
 		\begin{bmatrix}
	 		\nabla^2 f(x)  & A^\top\\
	 		A & 0 
 		\end{bmatrix} 
 		\begin{bmatrix}
	 		v\\
	 		w
 		\end{bmatrix} = 
 		\begin{bmatrix}
	 		-\nabla f(x)\\
	 		0
 		\end{bmatrix} 
 	$$
 	Как понять, что пора прекращать итерации? По сути, мы ищем оптимальное квадратичное приближение на допустимом множестве. Остаток запишется так: 
 	$$
		f(x) - \inf_{v} (\widehat{f}(x + v)\ |\ A(x + v) = b) 		
 	$$
 	где $ \widehat{f} $ — квадратичная аппроксимация функции $ f $.
 	Преобразуем одно из условий оптимальности.\\
 	Пусть $ h $ — направление убывания в точке $ x $. Тогда:
 	\begin{align*}
	 	\nabla^2 f(x) h + A^\top w &= -\nabla f(x)\ |\ h^{\top}\\
	 	h^\top \nabla^2 f(x) h + (Ah)\top w &= -h^\top \nabla f(x)\
	 	[Ah = 0]\\
	 	h^\top \nabla^2 f(x) h &= - \nabla f(x)^\top h
 	\end{align*}
 	$$
	 	\inf_{v} (\widehat{f}(x + v)\ |\ A(x + v) = b) = f(x) - \frac{1}{2} h^{\top} \nabla^2 f(x) h
	$$
 	Т.е. $ h^\top \nabla^2 f(x) h $ — наиболее разумный критерий остановки метода Ньютона в такой задаче.\\
 	\begin{theorem}[\textit{О скорости сходимости}]$  $\\
 		Скорость сходимости аналогична сходимости метода Ньютона в безусловной оптимизации.\\
 		Пусть выполнены следующие условия:
 		\begin{itemize}[$\diamond$]
 			\item Множество уровней $ S = \{ x \in D(f)\ |\ f(x) \le f(x_0), Ax = b \} $ замкнуто, а $ x_0 \in D(f),\ Ax_0 = b $.
 			\item $ \nabla^2 f $ липшицев на $ S $.
 			\item Норма обратной матрицы ККТ системы ограничена сверху.
 		\end{itemize}
	 	Тогда метод Ньютона сходится к паре $ (x^*, \mu^*) $ глобально линейно, но локально квадратично.
 	\end{theorem}
 	\noindent Если же начальная точка не лежит в допустимом множестве. Тогда ККТ можно записать так:
 	$$
 		\begin{bmatrix}
	 		\nabla^2 f(x) & A^\top\\
	 		A & 0
 		\end{bmatrix}
 		\begin{bmatrix}
	 		v\\
	 		w
 		\end{bmatrix} = -
 		\begin{bmatrix}
 			\nabla f(x)\\
 			Ax - b
 		\end{bmatrix}
 	$$
 	При таком подходе достаточно, чтобы $ x_0 \in D(f) $, а такую точку найти сильно проще.
	\subsection{Общая задача выпуклой оптимизации. Барьерные методы}
	\begin{align*}
		\min_{x \in \Real^n}&\ f_0(x)\\
		\text{s.t. }& f_i(x) \le 0\quad i = 1,\ldots,m\\
								& Ax = b
	\end{align*}
	где ограничения выпуклые, дважды непрерывно дифференцируемые, а $ A \in \Real^{p \times n},\ \mathrm{rk}(A) = p < n $.\\
	Будем предполагать, что условия Слейтера выполнены, т.е. имеется сильная двойственность.\\
	Условия ККТ:
	\begin{align*}
		&Ax^* = b,\ f_i(x^*) \le 0\\
		&\lambda_i^* f_i(x^*) = 0\\
		&\lambda^* \succeq 0\\
		&\nabla_{x} \mathcal{L}(x^*, \mu^*, \lambda^*) = \nabla f_0(x^*) + \sum_{i=1}^{m} \lambda_i^* \nabla f_i(x^*) + A^\top \mu^{*} = 0
	\end{align*}
	Ограничения типа неравенств можно перенести в целевую функцию, используя индикаторы:
	$$
		I_{-}(u) = \begin{cases}
			0, & u \le 0\\
			+\infty, & u > 0
		\end{cases}
	$$
	Целевая функция при этом перестаёт быть дифференцируемой, в связи с чем индикатор приближают барьерами — последовательностью выпуклых дважды дифференцируемых функций $ g(t, x) $, которая в пределе (при $ t \to 0 $) сходится к индикатору.
	\begin{align*}
		\min_{x}&\ f(x) + \sum_{i=1}^{m} -t\log (-f_i(x))\\
		\text{s.t. }& Ax = b
	\end{align*}
	Эту задачу уже можно решать, как описано в предыдущем разделе.\\
	Тем не менее, барьерный метод требует допустимой начальной точки. Отсюда возникает естественное разделение алгоритма на две фазы: в первой ищется начальное приближение, а во второй оно используется на итерации барьерного метода.\\
	Первую фазу можно реализовать, например, так:
	\begin{align*}
		\min_{x,s}&\ \sum_{i=1}^{m} s_i\\
		\text{s.t. }& f_i(x) \le s_i\\
								& Ax = b
	\end{align*}
\end{document}     